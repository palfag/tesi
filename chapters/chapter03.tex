\chapter{Individuazione tecnologica per il progetto}

\section{Confronto vendors}
Nel mercato sono presenti diversi tipi di soluzioni DLP. Ogni vendor propone il suo prodotto che fa 
concorrenza agli altri. Attraverso un'analisi di mercato sarà più semplice capire quali sono le funzionalità
fondamentali che contraddistinguono una soluzione DLP.

La seguente lista elenca alcuni dei principali prodotti presenti sul mercato:

\begin{itemize}
    \item Endpoint Protector by \textbf{CoSoSys}
    \item \textbf{Digital Guardian} Endpoint DLP
    \item \textbf{Symantec} Data Loss Prevention
    \item \textbf{Comodo} MyDLP
    \item \textbf{Forcepoint} Data Loss Prevention
    \item \textbf{SecureTrust} Data Loss Prevention
    \item \textbf{McAfee} Total Protection for Data Loss Prevention
    \item \textbf{Check Point} Data Loss Prevention
    \item \textbf{Safetica} Data Loss Prevention
  \end{itemize}

  Per lo studio di mercato prenderemo in causa i primi cinque della lista. 
  La tabella \ref{tabellaVendor} mostra similitudini e differenze delle caratteristiche possedute dai 
  principali prodotti DLP sul mercato.

  \begin{table}[htp]
    \centering
    \resizebox{\textwidth}{!}{%
    \begin{tabular}{|l|c|c|c|c|c|}
    \hline
     &
      \textbf{\begin{tabular}[c]{@{}c@{}}Endpoint \\ Protector\end{tabular}} &
      \textbf{\begin{tabular}[c]{@{}c@{}}Digital\\ Guardian\end{tabular}} &
      \textbf{Symantec} &
      \textbf{MyDLP} &
      \textbf{Forcepoint} \\ \hline
    \rowcolor[HTML]{EFEFEF} 
    {\color[HTML]{333333} \textbf{\begin{tabular}[c]{@{}l@{}}Dati in Transito\\ Network\end{tabular}}} &
      {\color[HTML]{333333} } &
      {\color[HTML]{333333} } &
      {\color[HTML]{333333} } &
      {\color[HTML]{333333} } &
      {\color[HTML]{333333} } \\ \hline
    Traffico Web                                                                                                 & x & x & x & x & x \\ \hline
    Traffico Email                                                                                               & x & x & x & x & x \\ \hline
    Traffico IM                                                                                                  & x &   & x & x & x \\ \hline
    \rowcolor[HTML]{EFEFEF} 
    \textbf{\begin{tabular}[c]{@{}l@{}}Dati in Uso\\ Endpoint\end{tabular}}                                      &   &   &   &   &   \\ \hline
    \begin{tabular}[c]{@{}l@{}}Controllo Device\\ (USB, HDD, \\ SSD,CD/DVD, \\ fax, stampanti)\end{tabular}      & x & x & x & x & x \\ \hline
    Screenshots                                                                                                  & x & x & x & x & x \\ \hline
    Clipboard                                                                                                    & x & x & x & x & x \\ \hline
    \begin{tabular}[c]{@{}l@{}}Controllo \\ applicazioni\end{tabular}                                            &   & x & x &   & x \\ \hline
    \begin{tabular}[c]{@{}l@{}}Supporto ai \\ dispositivi mobili\\ (telefoni, tablet)\end{tabular}               &   & x & x &   & x \\ \hline
    \begin{tabular}[c]{@{}l@{}}Supporti di \\ virtualizzazione\\ (Virtualbox, VMware)\end{tabular}               & x & x & x &   & x \\ \hline
    \rowcolor[HTML]{EFEFEF} 
    \textbf{\begin{tabular}[c]{@{}l@{}}Dati a riposo\\ Discovery\end{tabular}}                                   &   &   &   &   &   \\ \hline
    Endpoint discovery                                                                                           & x & x & x & x & x \\ \hline
    Database discovery                                                                                           &   &   & x & x & x \\ \hline
    \rowcolor[HTML]{EFEFEF} 
    \textbf{\begin{tabular}[c]{@{}l@{}}Metodi di \\ Content detection\end{tabular}}                              &   &   &   &   &   \\ \hline
    Espressioni regolari                                                                                         & x & x & x & x & x \\ \hline
    \begin{tabular}[c]{@{}l@{}}OCR\\ Optical character\\ recognition\end{tabular}                                & x &   & x &   & x \\ \hline
    \begin{tabular}[c]{@{}l@{}}PDM\\ partial document matching\end{tabular}                                      &   & x & x & x & x \\ \hline
    \begin{tabular}[c]{@{}l@{}}EDM\\ exact document matching\\ (fingerprinting di dati strutturati)\end{tabular} & x & x & x & x & x \\ \hline
    Parole chiave - Dizionario                                                                                   & x & x & x & x & x \\ \hline
    Analisi del contesto                                                                                         &   & x & x &   & x \\ \hline
    \begin{tabular}[c]{@{}l@{}}Fingerprinting \\ di dati non strutturati\end{tabular}                            & x & x & x & x & x \\ \hline
    \textbf{Gestione gerarchica delle policy}                                                                    & x &   & x &   & x \\ \hline
    \textbf{Open Source}                                                                                         &   &   &   &   &   \\ \hline
    \end{tabular}%
    }
    \caption{Confronto prodotti DLP}\label{tabellaVendor}
    \end{table}


    \section{Soluzioni open source}
    Come si evince nella tabella \ref*{tabellaVendor} si può facilmente notare che nessuno dei 
    prodotti è open source. Perché? Nel mercato non si trovano tecnologie DLP gratuite valide da
    poter tenere testa a quelle proprietarie. La maggior parte dei prodotti DLP sono a pagamento e 
    sono accompagnati da una demo in modo da poter essere testati.

    \textit{OpenDLP} è un prodotto DLP open source, disponibile per sistemi Windows e Unix. Purtroppo
    non è più mantenuto, infatti l'ultimo aggiornamento è datato 2012. Vale la pena citarlo perché non 
    vi sono altre soluzioni DLP gratuite. OpenDLP può essere scaricato al sito: 
    \url{https://code.google.com/archive/p/opendlp/}.
    
    Un altro prodotto degno di nota è \textit{MyDLP}. Nato come software open source,
    ma dopo l'acquisizione da parte di Comodo nel 2014, la versione aperta presente su github
    \url{https://github.com/mydlp} non è stata più aggiornata.

    \begin{figure}
        \centering
        \fbox{\includegraphics[width=.8\textwidth, height=.6\textheight, keepaspectratio]{data loss vectors.jpeg}}
        \caption{Modello ISACA: Data Loss Vectors}\label{ModelloIsaca}
      \end{figure}

\section{Accenni sul progetto}
Il progetto implementato consiste in una soluzione DLP per la protezione dei dati in transito, più precisamente per
l'identificazione di eventuali contenuti sensibili presenti nelle email aziendali, in modo da prevenirne la perdita
ed evitarne la diffusione.

Il confronto effettuato tra i prodotti principali presenti sul mercato, anche se nessuno di essi è stato utilizzato 
ai fini del progetto, è stato utile per capire quali caratteristiche deve possedere la soluzione DLP implementata.

\section{Tecnologie utilizzate}
Per l'implementazione del progetto sono state utilizzate diverse tecnologie:
    \subsection{Linux}
    \subsection{Postfix}
    \subsection{Python}
    \subsection{Bash}
    \subsection{Apache Tika}
    \subsection{Virtualbox}

\section{Protocolli utilizzati}
    \subsection{SSH}
    \subsection{SMTP}
    \subsection{TLS/SSL}