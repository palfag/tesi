\chapter{Analisi e Disegno}


Questo capitolo ha come obiettivo quello di descrivere tutte le funzionalità che deve soddisfare la 
soluzione DLP e gli eventuali vincoli a cui deve attenersi.

\section{Soluzione DLP per il traffico email}
Tenuto conto del modello ISACA,

La soluzione DLP da implementare consisterà nella protezione dei data-in-motion, più nello specifico
andrà a focalizzarsi sulla protezione delle email. La soluzione andrà a monitorare il traffico email e
il loro contenuto in modo da:

\begin{enumerate}
    \item {identificare la presenza di dati sensibili al loro interno;}
    \item {prevenire la loro perdita / 
    invio (in modo da prevenire la fuoriuscita dall’ambiente aziendali di queste informazioni).}
\end{enumerate}

\section{Motivazione scelta e obiettivo}
La scelta di implementare questa funzionalità è dovuta al fatto che le email sono uno, 
se non il principale, dei metodi di comunicazione in ambito aziendale.
L’implementazione di questa funzionalità, oltre ad essere utile, risulta anche interessante.

Lo scopo principale è quello di incrementare la sicurezza enterprise, quindi in contesto aziendale, 
evitando la diffusione di dati sensibili.

\pagebreak
\section{Glossario}

\begin{table}[!b]
    \centering
    \resizebox{\textwidth}{!}{%
    \begin{tabular}{|l|l|}
    \hline
    \rowcolor[HTML]{EFEFEF} 
    \textbf{Termine} &
      \textbf{Definizione} \\ \hline 
    Dipendente &
      \begin{tabular}[c]{@{}l@{}}Membro del personale. \\ Può essere il mittente o \\ il destinatario di una email.\end{tabular} \\ \hline 
    Mittente &
      \begin{tabular}[c]{@{}l@{}}Colui che invia la email.\\ Per questo progetto il mittente \\ è inteso come un dipendente \\ interno all’azienda.\end{tabular} \\ \hline
    Destinatario &
      \begin{tabular}[c]{@{}l@{}}Colui che riceve la email.\\ Può essere sia interno che \\ esterno all’azienda.\end{tabular} \\ \hline
    Email &
      \begin{tabular}[c]{@{}l@{}}Oggetto che va monitorato e \\ ispezionato dalla soluzione DLP.\\ In generale una email contiene:\\ -  Mittente;\\ -  Destinatario;\\ -  Oggetto;\\ -  Corpo;\\ -  Allegato (opzionale).\\ \\ I primi due attributi possono \\ essere oggetto di analisi del contesto, \\ mentre gli ultimi tre verranno ispezionati \\ durante l’analisi del contenuto.\end{tabular} \\ \hline
    Dati sensibili &
      \begin{tabular}[c]{@{}l@{}}Dati che hanno maggior valore, ad esempio:\\ - dati di business\\  (contratti, proprietà intellettuale, progetti);\\  \\ - dati personali \\ (Codici fiscali, date di nascita \\ e altre informazioni sui dipendenti \\ e/o clienti);\\ \\ - dati finanziari \\ (dati della carta di credito o di altre modalità di pagamento);\\ eccetera.\end{tabular} \\ \hline
    \end{tabular}%
    }
    \caption{Glossario dei termini}\label{Glossario}
    \end{table}

    \pagebreak
    \section{Analisi dei requisiti}

    \subsection*{Obiettivo utente:}
    La soluzione DLP deve identificare la presenza di dati/documenti sensibili all’interno delle email 
    in modo da evitare che lascino il sistema aziendale.

    \subsection*{Requisiti funzionali:}



\begin{table}[htp]
    \centering
    \resizebox{\textwidth}{!}{%
    \begin{tabular}{ll}
    \multicolumn{2}{l}{requisito funzionale 1 (RF01): \textbf{Monitoraggio traffico SMTP}}                               \\ \hline
    \rowcolor[HTML]{EFEFEF} 
    \multicolumn{1}{|l|}{\cellcolor[HTML]{EFEFEF}ID}       & \multicolumn{1}{l|}{\cellcolor[HTML]{EFEFEF}RF01}           \\ \hline
    \multicolumn{1}{|l|}{Nome}                             & \multicolumn{1}{l|}{Monitoraggio traffico email (SMTP)}     \\ \hline
    \rowcolor[HTML]{EFEFEF} 
    \multicolumn{1}{|l|}{\cellcolor[HTML]{EFEFEF}Definizione} &
      \multicolumn{1}{l|}{\cellcolor[HTML]{EFEFEF}La soluzione deve monitorare il traffico email} \\ \hline
    \multicolumn{1}{|l|}{Motivazione} &
      \multicolumn{1}{l|}{\begin{tabular}[c]{@{}l@{}}Senza l'implementazione di questo requisito,\\ la soluzione non può funzionare. È il requisito\\ base per poter implementare gli altri\end{tabular}} \\ \hline
    \rowcolor[HTML]{EFEFEF} 
    \multicolumn{1}{|l|}{\cellcolor[HTML]{EFEFEF}Priorità} & \multicolumn{1}{l|}{\cellcolor[HTML]{EFEFEF}Indispensabile} \\ \hline
    \multicolumn{1}{|l|}{Dipendenze}                       & \multicolumn{1}{l|}{/}                                      \\ \hline
    \end{tabular}%
    }
    \newline
    \vspace*{1 cm}
    \newline
\end{table}



\begin{table}[htp]
    \centering
    \resizebox{\textwidth}{!}{%
    \begin{tabular}{ll}
    \multicolumn{2}{l}{requisito funzionale 2 (RF02): \textbf{Analisi del contenuto}}                                    \\ \hline
    \rowcolor[HTML]{EFEFEF} 
    \multicolumn{1}{|l|}{\cellcolor[HTML]{EFEFEF}ID}       & \multicolumn{1}{l|}{\cellcolor[HTML]{EFEFEF}RF02}           \\ \hline
    \multicolumn{1}{|l|}{Nome}                             & \multicolumn{1}{l|}{Analisi del contenuto}                  \\ \hline
    \rowcolor[HTML]{EFEFEF} 
    \multicolumn{1}{|l|}{\cellcolor[HTML]{EFEFEF}Definizione} &
      \multicolumn{1}{l|}{\cellcolor[HTML]{EFEFEF}\begin{tabular}[c]{@{}l@{}}La soluzione deve analizzare il contenuto di\\ ogni email che viene inviata.\end{tabular}} \\ \hline
    \multicolumn{1}{|l|}{Motivazione} &
      \multicolumn{1}{l|}{\begin{tabular}[c]{@{}l@{}}Senza l'implementazione di questo requisito,\\ La soluzione non sarebbe in grado di\\ \\ identificare contenuti sensibili\end{tabular}} \\ \hline
    \rowcolor[HTML]{EFEFEF} 
    \multicolumn{1}{|l|}{\cellcolor[HTML]{EFEFEF}Priorità} & \multicolumn{1}{l|}{\cellcolor[HTML]{EFEFEF}Indispensabile} \\ \hline
    \multicolumn{1}{|l|}{Dipendenze}                       & \multicolumn{1}{l|}{RF01}                                   \\ \hline
    \end{tabular}%
    }
    \end{table}


\begin{table}[htp]
    \centering
    \resizebox{\textwidth}{!}{%
    \begin{tabular}{ll}
    \multicolumn{2}{l}{requisito funzionale 3 (RF03): \textbf{Analisi del contesto}}                                                   \\ \hline
    \rowcolor[HTML]{EFEFEF} 
    \multicolumn{1}{|l|}{\cellcolor[HTML]{EFEFEF}ID}       & \multicolumn{1}{l|}{\cellcolor[HTML]{EFEFEF}RF03}                         \\ \hline
    \multicolumn{1}{|l|}{Nome}                             & \multicolumn{1}{l|}{Analisi del contesto}                                 \\ \hline
    \rowcolor[HTML]{EFEFEF} 
    \multicolumn{1}{|l|}{\cellcolor[HTML]{EFEFEF}Definizione} &
      \multicolumn{1}{l|}{\cellcolor[HTML]{EFEFEF}\begin{tabular}[c]{@{}l@{}}La soluzione, oltre ad analizzare il contenuto\\ di una email, deve anche tenere conto del \\ contesto\end{tabular}} \\ \hline
    \multicolumn{1}{|l|}{Motivazione} &
      \multicolumn{1}{l|}{\begin{tabular}[c]{@{}l@{}}Effettuare l'analisi del contesto può essere un\\ utile supporto all'analisi del contenuto.\end{tabular}} \\ \hline
    \rowcolor[HTML]{EFEFEF} 
    \multicolumn{1}{|l|}{\cellcolor[HTML]{EFEFEF}Priorità} & \multicolumn{1}{l|}{\cellcolor[HTML]{EFEFEF}Non Indispensabile, ma utile} \\ \hline
    \multicolumn{1}{|l|}{Dipendenze}                       & \multicolumn{1}{l|}{RF01}                                                 \\ \hline
    \end{tabular}%
    }
    \end{table}



\begin{table}[htp]
    \centering
    \resizebox{\textwidth}{!}{%
    \begin{tabular}{ll}
    \multicolumn{2}{l}{requisito funzionale 4 (RF04): \textbf{Identificazione dati sensibili}}                   \\ \hline
    \rowcolor[HTML]{EFEFEF} 
    \multicolumn{1}{|l|}{\cellcolor[HTML]{EFEFEF}ID}       & \multicolumn{1}{l|}{\cellcolor[HTML]{EFEFEF}RF04}           \\ \hline
    \multicolumn{1}{|l|}{Nome}                             & \multicolumn{1}{l|}{Identificazione di contenuti sensibili} \\ \hline
    \rowcolor[HTML]{EFEFEF} 
    \multicolumn{1}{|l|}{\cellcolor[HTML]{EFEFEF}Definizione} &
      \multicolumn{1}{l|}{\cellcolor[HTML]{EFEFEF}\begin{tabular}[c]{@{}l@{}}La soluzione dev'essere in grado di identificare eventuali\\ dati/contenuti sensibili presenti all'interno delle email\end{tabular}} \\ \hline
    \multicolumn{1}{|l|}{Motivazione} &
      \multicolumn{1}{l|}{\begin{tabular}[c]{@{}l@{}}Senza l'implementazione di questo requisito non è \\ possibile raggiungere l'obiettivo utente\end{tabular}} \\ \hline
    \rowcolor[HTML]{EFEFEF} 
    \multicolumn{1}{|l|}{\cellcolor[HTML]{EFEFEF}Priorità} & \multicolumn{1}{l|}{\cellcolor[HTML]{EFEFEF}Indispensabile} \\ \hline
    \multicolumn{1}{|l|}{Dipendenze}                       & \multicolumn{1}{l|}{RF01, RF02}                             \\ \hline
    \end{tabular}%
    }
    \end{table}


\begin{table}[htp]
    \centering
    \resizebox{\textwidth}{!}{%
    \begin{tabular}{ll}
    \multicolumn{2}{l}{requisito funzionale 5 (RF05): \textbf{Azioni di risposta}}                         \\ \hline
    \rowcolor[HTML]{EFEFEF} 
    \multicolumn{1}{|l|}{\cellcolor[HTML]{EFEFEF}ID}       & \multicolumn{1}{l|}{\cellcolor[HTML]{EFEFEF}RF05}           \\ \hline
    \multicolumn{1}{|l|}{Nome}                             & \multicolumn{1}{l|}{Intraprendere azioni di risposta}       \\ \hline
    \rowcolor[HTML]{EFEFEF} 
    \multicolumn{1}{|l|}{\cellcolor[HTML]{EFEFEF}Definizione} &
      \multicolumn{1}{l|}{\cellcolor[HTML]{EFEFEF}\begin{tabular}[c]{@{}l@{}}La soluzione deve poter intraprendere una varietà di azioni:\\ 1. consentire l’invio;\\ 2. bloccare l’invio;\\ 3. eliminare il contenuto sensibile e procedere con l’invio.\end{tabular}} \\ \hline
    \multicolumn{1}{|l|}{Motivazione} &
      \multicolumn{1}{l|}{\begin{tabular}[c]{@{}l@{}}Poiché la soluzione deve evitare che documenti sensibili \\ lascino il sistema aziendale, l'implementazione di questo \\ requisito è fondamentale\end{tabular}} \\ \hline
    \rowcolor[HTML]{EFEFEF} 
    \multicolumn{1}{|l|}{\cellcolor[HTML]{EFEFEF}Priorità} & \multicolumn{1}{l|}{\cellcolor[HTML]{EFEFEF}Indispensabile} \\ \hline
    \multicolumn{1}{|l|}{Dipendenze}                       & \multicolumn{1}{l|}{RF01, RF04}                             \\ \hline
    \end{tabular}%
    }
    \end{table}


\begin{table}[htp]
\centering
\resizebox{\textwidth}{!}{%
\begin{tabular}{ll}
\multicolumn{2}{l}{requisito funzionale 6 (RF06): \textbf{Avviso in caso di blocco}}                                 \\ \hline
\rowcolor[HTML]{EFEFEF} 
\multicolumn{1}{|l|}{\cellcolor[HTML]{EFEFEF}ID}       & \multicolumn{1}{l|}{\cellcolor[HTML]{EFEFEF}RF06}           \\ \hline
\multicolumn{1}{|l|}{Nome}                             & \multicolumn{1}{l|}{Avviso in caso di blocco}               \\ \hline
\rowcolor[HTML]{EFEFEF} 
\multicolumn{1}{|l|}{\cellcolor[HTML]{EFEFEF}Definizione} &
  \multicolumn{1}{l|}{\cellcolor[HTML]{EFEFEF}\begin{tabular}[c]{@{}l@{}}La soluzione deve sempre avvisare il dipendente, e \\ l'analista DLP in caso decida di bloccare l'invio della \\ email\end{tabular}} \\ \hline
\multicolumn{1}{|l|}{Motivazione} &
  \multicolumn{1}{l|}{\begin{tabular}[c]{@{}l@{}}Utile avvisare il dipendente (e l'analista DLP) in caso\\ sia necessario bloccare l'invio della sua email\end{tabular}} \\ \hline
\rowcolor[HTML]{EFEFEF} 
\multicolumn{1}{|l|}{\cellcolor[HTML]{EFEFEF}Priorità} & \multicolumn{1}{l|}{\cellcolor[HTML]{EFEFEF}Indispensabile} \\ \hline
\multicolumn{1}{|l|}{Dipendenze}                       & \multicolumn{1}{l|}{RF05}                                   \\ \hline
\end{tabular}%
}
\newline
\vspace*{1 cm}
\newline
\end{table}

\pagebreak
\section{Dipendenze tra requisiti}
Di seguito ho stilato un elenco delle dipendenze formatesi tra i requisiti.
La nomenclatura utilizzata è la seguente:
\begin{flushleft}
  x -> y: il requisito x dipende dal requisito y
  \newline
  \newline
  RF02 -> RF01

  RF03 -> RF01

  RF04 -> RF01, RF02

  RF05 -> RF01, RF04

  RF06 -> RF05
  \newline
  \newline
  Elenco delle dipendenze semplificato: 
  \newline
  \newline
  RF02 -> RF01

  RF03 -> RF01

  RF04 -> RF02

  RF05 -> RF04 
  
  RF06 -> RF05
\end{flushleft}

\newpage
\section{Grafo delle Dipendenze}
Le dipendenze tra i requisiti formano un grafo aciclico che bisogna tenere in considerazione in caso si debba modificare uno dei requisiti del sistema.

\begin{figure}[htp]
  \centering
  \includegraphics[width=.4\textwidth, height=.3\textheight, keepaspectratio]{grafo.jpg}
  \caption{Grafo aciclico delle dipendenze}\label{grafoDipendenze}
\end{figure}

\section{Disegno dell'architettura}

Per la progettazione della soluzione inseriremo la nostra soluzione DLP
tra il mail client (Outlook) e il mail server aziendale (Aruba). La soluzione DLP deciderà se lasciar 
passare l' email verso il mail server aziendale, oppure bloccarne l'invio.
\begin{figure}[htp]
  \centering
  \includegraphics[width=12cm, height=16cm, keepaspectratio]{disegno1.png}
  \caption{disegno architettura per un mail client}\label{disegno1}
\end{figure}

\pagebreak
La soluzione DLP verrà implementata attraverso l'uso di un mail server di inoltro (relay mail server).
Il disegno subirà una piccola modifica.

\begin{figure}[htp]
  \centering
  \includegraphics[width=12cm, height=16cm, keepaspectratio]{disegno2.png}
  \caption{soluzione DLP attraverso mail server di relay}\label{disegno2}
\end{figure}

Come ultimo passo andremo a considerare una sottorete composta dagli host dei dipendenti dell'azienda, al
posto di un unico client. Questa è la configurazione che più rispecchia la realtà.

\begin{figure}[htp]
  \centering
  \includegraphics[width=12cm, height=16cm, keepaspectratio]{disegno3.png}
  \caption{soluzione DLP attraverso mail server di relay}\label{disegno3}
\end{figure}

\subsection{Descrizione del flusso di dati e di controllo}
Come mostrato dalla figure \ref{disegno2} e \ref{disegno3}, Quando un dipendente invia una email, questa
passerà dal relay mail server che deciderà se bloccarla, oppure inoltrarla al mail server aziendale. L'email
continuerà il suo cammino fino ad arrivare al mail server del destinatario. A quel punto potrà essere scaricata
dal mail client del destinatario.