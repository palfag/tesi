\chapter{Introduzione}

Nell'ultimo anno si è rilevato un profondo cambiamento riguardante le abitudini della società.
Il COVID-19 ha impattato il mondo costringendolo a modificarsi e con se ha stravolto anche il modo di lavorare. 
Le restrizioni imposte dai governi hanno incoraggiato i dipendenti a lavorare da casa.
L'aumento del lavoro a distanza ha portato con se un aumento degli attacchi informatici. 
Durante il lockdown si è registrato che più di mezzo milione di persone in tutto il mondo sono state 
vittima di violazioni in cui i loro dati sono stati rubati e venduti sul dark web.\cite{Intro}

Anche le aziende sono state vittima di numerosi attacchi. Nel settore aziendale è impensabile 
non proteggersi dal punto di vista informatico, quindi, accanto alle protezioni perimetrali classiche 
per impedire attacchi dall’esterno, costituite da Firewall, antivirus, antispam, anti-malware, 
sistemi di prevenzione delle intrusioni (IPS) e sistemi di controllo degli accessi di rete (NAC) 
è assolutamente necessario prevedere anche sistemi che impediscano alle informazioni di uscire 
dai sistemi aziendali: i cosiddetti sistemi di Data Loss Prevention (DLP).

Questa tesi documenta tutte le fasi progettuali, dalla rilevazione dei requisiti, alla progettazione, 
l’implementazione, fino al collaudo e al Go Live di un sistema DLP in grado di identificare dati riservati, 
contenuti all'interno delle email, così da prevenirne la perdita ed evitarne la diffusione.

\section{Struttura della tesi}

Il capitolo 1 tratta una breve introduzione al lavoro di tirocinio, 
spiegando perché è importante prevedere sistemi di Data Loss Prevention, 
viene anche fornita una panoramica della struttura di questa tesi.

Il capitolo 2 fornisce le basi teoriche riguardo la tematica DLP e 
della posta elettronica, in modo da poter comprendere al meglio i capitoli successivi.

Nel capitolo 3 viene effettuato un confronto tra i principali prodotti presenti sul mercato, 
in modo da capire quali sono le caratteristiche fondamentali che contraddistinguono una soluzione DLP; 
sempre nel capitolo 3 viene effettuata una piccola introduzione sul progetto sviluppato e sulle tecnologie 
utilizzate per la sua implementazione.

Il capitolo 4 affronta le fasi di analisi e disegno dell'architettura necessarie 
per la successiva implementazione della soluzione.

Il capitolo 5 è interamente incentrato su Postfix, viene descritto il suo funzionamento, 
la sua architettura e come è stato utilizzato per soddisfare i requisiti scoperti durante la fase di analisi.

Nel capitolo 6 viene viene descritta l'installazione e la configurazione di Postfix in modo che si interfacci con
il mail client utilizzato in azienda Outlook e che svolga la funzione di mail server di inoltro verso il servizio
di posta elettronica aziendale di Aruba.
Per garantire la sicurezza, il traffico viaggerà utilizzando canali di comunicazione sicuri.
Viene inoltre descritto lo sviluppo di alcuni filtri interni, per l'analisi del contenuto e del contesto,
attraverso l'uso di parole chiave e di espressioni regolari;
sempre nel capitolo 6 è descritto lo sviluppo di uno script Python per effettuare l'analisi del contenuto 
degli allegati 
che successivamente è stato integrato in Postfix. Segue la descrizione della fase di test. 

Infine un ultimo spazio è dedicato alle conclusioni e ai possibili sviluppi futuri.




