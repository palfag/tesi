\chapter{DLP e posta elettronica}

In questo capitolo si riportano gli aspetti teorici principali riguardanti la tematica della Data Loss Prevention
e viene descritto il funzionamento della posta elettronica, in modo da comprendere al meglio i capitoli
successivi.

\section{Data Loss Prevention}
Con il termine Data Loss Prevention si fa riferimento a 
tecniche e sistemi in grado di identificare, monitorare 
e proteggere dati riservati con l’obiettivo di individuare 
e prevenire l’uso non autorizzato e la diffusione di informazioni riservate \cite{DLP1}. 

Vi sono diversi tipi di minacce che possono portare ad una perdita di dati.
La perdita di dati, in un'azienda, può avvenire accidentalmente, per negligenza di un 
dipendente. È possibile, che a provocare una perdita di dati, sia un dipendente malintenzionato,
mosso dall'intenzione di danneggiare l'azienda o di rubare informazioni al suo datore di lavoro.
Vi è anche la possibilità che un hacker, attraverso tecniche di ingegneria sociale, riesca a rubare
le credenziali di un dipendente e ad intrufolarsi all'interno della rete aziendale, ovviamente con 
l'obiettivo di appropriarsi di dati riservati.
Queste sono alcune delle possibili vie che possono causare una perdita di dati. Poiché una perdita di dati
può comportare sanzioni, conseguenze legali, ingenti perdite economiche, fino al compromettere la reputazione del 
marchio e portare al fallimento dell'azienda, è necessaria l'implementazione di soluzioni
DLP\footnote{\textit{DLP}: Data Loss Prevention}.
 

\subsection{Cosa fa una soluzione DLP}
    Una soluzione DLP monitora il traffico dei dati all'interno di un'azienda per accorgersi se
    questi circolano impropriamente. In caso di anomalie, vengono attivate delle azioni di risposta 
    in modo da bloccare il trasferimento dati non autorizzato e prevenire quindi la loro perdita/diffusione \cite{DLP2}.
    


\subsection{Tipi di dati monitorati}
    Sostenuto che il dato è di fondamentale importanza e che le soluzioni DLP si occupano di proteggere i dati riservati,
    ora, è opportuno spiegare cosa s'intende quando si parla di dati riservati e che tipi di dati vengono monitorati 
    da una soluzione DLP. I dati riservati sono dati che hanno maggior valore per un'azienda e che possono comportare 
    delle conseguenze se vengono persi o divulgati.

    In natura esistono due tipi dati: \textit{dati strutturati e non strutturati}.
        \subsubsection{dati strutturati}
            I dati strutturati sono dati che, come dice il nome, hanno una struttura, che presentano un pattern comune, e che 
            quindi possono essere facilmente individuati tramite le espressioni regolari. 
            Sono esempi di dati strutturati numeri delle carte di credito, codici fiscali, eccetera.
            Il 20\% dei dati sono di tipo strutturato. 
    
            \begin{center} 
                \textit{Espressione regolare per carte di credito Visa Mastercard:}
                \verb/^(?:4[0-9]{12}(?:[0-9]{3})?|5[1-5][0-9]{14})$/ 
            \end{center}

        \subsubsection{dati non strutturati}
            I dati non strutturati, a differenza dei precedenti, non presentano un modello ripetibile
            e quindi prevedibile. L'80\% di tutti i dati rientra in questa categoria. Esempi di dati non 
            strutturati sono email, immagini, audio, video, file pdf, eccetera. 
    
    Le espressioni regolari non sono adatte per identificare questo tipo di dati. %(possono essere usate per l'estrazione di informazioni strutturate da dati testuali non strutturati), 
    Per i dati non strutturati si usano tecniche differenti, più sofisticate, alcune delle quali saranno introdotte nel paragrafo successivo. \cite{DLP3}.

    \subsubsection*{}
    I dati, durante la loro vita, possono trovarsi a riposo, in utilizzo o in transito.
    L'obiettivo della DLP è quello di proteggere i dati nei diversi stati in cui si possono trovare:
    \cite{DLP1}

    \subsubsection{data-at-rest}
        I dati a riposo sono dati al momento inattivi, che sono immagazzinati fisicamente in database,
        o in file (ad esempio in file excel), oppure su dispositivi di storage come HDD o SSD, eccetera.
        Se sono salvati in un computer, questi di dati si trovano in memoria secondaria.
        Le modalità di protezione di questi dati sono denominate DARP\footnote{\textit{DARP}: Data At Rest Protection}, 
        di cui fanno parte tecniche come la crittografia dei dati (prima di essere archiviati) oppure come la cifratura
        dell'intero dispositivo di storage, protezione con password, il controllo degli accessi, eccetera.

    \subsubsection{data-in-use}
        A differenza dei dati a riposo, questi sono dati attivi, ovvero dati che stanno venendo
        visualizzati o manipolati da un'applicazione, dati con cui un utente sta interagendo. 
        Questi dati si differenziano dai dati a riposo perché, essendo al momento in utilizzo, 
        si troveranno in qualche registro della CPU, in qualche cache 
        o comunque in memoria principale. Sono più complessi da proteggere perché, dato che l'utente vi sta
        interagendo, si troveranno necessariamente in chiaro. Per la protezione di questi tipi di dati
        vengono usate tecniche che bloccano l'uso delle porte fisiche, il copia e incolla, gli screenshots, l'uso dei fax
        e delle stampanti.

    \subsubsection{data-in-motion}
        Sono dati in transito quei dati che al momento stanno attraversando una rete per raggiungere 
        una destinazione finale. Le misure di protezione dei dati in transito sono fondamentali in quanto
        i dati sono considerati meno sicuri durante il movimento.
        Alcune tecniche di protezione base sono la cifratura dei dati
        e l'utilizzo di connessioni sicure (HTTPS, FTPS, TLS/SSL, SMTPS).

  %      \begin{figure}[htp]
    %         \vspace{0.5cm}
    %        \centering
     %       \fbox{\includegraphics[width=.8\textwidth, height=.6\textheight, keepaspectratio]{data loss vectors.jpeg}}
      %      \caption{Modello ISACA: Data Loss Vectors}\label{ModelloIsaca}
       %   \end{figure}
%\pagebreak
\subsection{Identificazione dei dati}
    Vi sono diversi metodi per l'identificazione di dati riservati, l'utilizzo di una tecnica piuttosto che
    un'altra dipende se il dato è strutturato o meno. Di seguito 
    una breve spiegazione su alcune delle tecniche per identificare dati riservati: \cite{DLP4}
    \subsubsection{Pattern matching}
    Questa tecnica si basa sull'utilizzo delle espressioni regolari che hanno come scopo quello
    di identificare pattern prestabiliti. Come detto in precedenza è la tecnica maggiormente 
    utilizzata per i dati strutturati. 

    \subsubsection{Dictionary lookup}
    La tecnica del dizionario si basa sull'utilizzo di un file di testo che ha come funzione quella
    di un vero e proprio dizionario. All'interno, il file viene riempito di parole chiave. 
    Il dizionario viene utilizzato come database di confronto per identificare dati riservati all'interno 
    di documenti. Questo metodo è particolarmente adatto per dati non strutturati.

    \subsubsection{File fingerprinting}
    Anche questa tecnica è utilizzata per dati non strutturati come file, immagini, eccetera.
    Attraverso questa tecnica si può confrontare l'impronta di un file inviato via mail con 
    le impronte di file riservati precedentemente calcolate, salvate in database.
    Se ad esempio l'impronta del file in transito coincide con una di quelle del database,
    allora il file è un file riservato e si deve bloccare la sua diffusione.

\subsection{Network DLP vs Endpoint DLP}
    Nel mercato sono presenti diversi tipi di prodotti DLP, ognuno dei quali si concentra sulla protezione
    di un diverso tipo di dato. I principali sono i network DLP che si concentrano sulla protezione dei dati
    in transito, e gli endpoint DLP che invece sono incentrati sulla protezione dei dati in uso.
    Non vi è un prodotto migliore dell'altro perché ognuno presenta dei vantaggi e non è esente da difetti.

    \subsubsection{Network DLP}
            È inserito all'interno di una rete. Può comportare un minimo di overhead poiché tutto il traffico
            passa attraverso il dispositivo DLP per essere analizzato.
            Il network DLP monitora i dati che attraversano la rete e applica le politiche che sono in vigore
            in quel momento. Quando ad esempio un utente prova ad inviare un file riservato utilizzando la posta
            elettronica, il  dispositivo nDLP ispeziona il traffico e attraverso criteri predefiniti 
            può bloccare, mettere in quarantena o crittografare la email.
            Uno svantaggio di questo dispositivo è che se il device non si trova all'interno della rete aziendale,
            e non viene utilizzata una corporate VPN, il nDLP non ha visibilità su cosa stia succedendo con quei 
            dati.

    \subsubsection{Endpoint DLP}
            Consiste in un agent che va installato in ogni computer che si vuole monitorare. A differenza del precedente 
            però l'eDLP protegge sempre i dati, anche quando il computer non è all'interno della rete aziendale, ma ad esempio in un 
            aeroporto, utilizzando una rete pubblica. Un endpoint DLP permette di implementare le tecniche di sicurezza
            citate precedentemente per la protezione dei dati in uso. un eDLP può ad esempio proteggere file riservati
            dall'essere copiati su pendrive o HDD esterni.
    
    \subsubsection*{}     
    Essendo prodotti con obiettivi diversi, è consigliato che le aziende utilizzino una soluzione che comprenda entrambi
    in modo da essere protette su più fronti \cite{DLP5}.


\pagebreak
\section{Posta elettronica}
La posta elettronica è uno dei servizi Internet più utilizzati in tutto il mondo\cite{posta}. Attraverso di essa
È possibile scambiare messaggi attraverso la rete in modo facile, veloce e soprattutto gratuito. 
I messaggi scambiati possono consistere in semplice testo (oggetto e corpo), 
oppure includere anche uno o più allegati.

\subsection{Formato dei messaggi di posta elettronica}
Un messaggio di posta elettronica contiene una serie di righe di intestazione (header).
Ogni intestazione è una coppia chiave:valore. Alcune parole chiave sono opzionali (Subject:), 
ma altre sono obbligatorie come From: e To:. Una riga vuota separa gli header dal corpo del messaggio.
Di seguito un esempio di messaggio di posta:

\begin{verbatim}
    User-Agent: Microsoft-MacOutlook/16.49.21050901
    Date: Wed, 19 May 2021 12:12:08 +0200
    Subject: Importante
    From: Paolo Fagioli <paolo.fagioli@certimeter.it>
    To: Paolo Fagioli <palfag33@gmail.com>
    Message-ID: <9C8047BB-21D6-4D68-B33E-CCF3B68399DB@certimeter.it>
    Thread-Topic: Importante
    Mime-version: 1.0
    Content-type: text/plain;
        charset="UTF-8"
    Content-transfer-encoding: quoted-printable

    Invio un messaggio di prova.
    Saluti,

    Paolo Fagioli
\end{verbatim}

\subsection{Architettura del sistema di posta elettronica}
La posta elettronica è basata su un'architettura client-server. 
Sia lato mittente che lato ricevente, viene utilizzata un'applicazione, lo user agent, per rispettivamente 
comporre ed inviare, e leggere i messaggi ricevuti. 
Alcuni esempi di user agent sono Microsoft Outlook, utilizzato per lo sviluppo della soluzione DLP, 
e Mozilla Thunderbird.
Un altro componente fondamentale per il funzionamento della posta elettronica è il mail server. 
Un mail server svolge principalmente due funzioni:

\begin{enumerate}
    \item salva il messaggio nella casella di posta dell'utente (è importante sapere che ogni utente ha una casella di posta salvata in un qualche mail server);
    \item inoltra i messaggi di posta verso un altro mail server.
\end{enumerate}

Il protocollo principale utilizzato per lo scambio di messaggi di posta elettronica è SMTP, 
che viene affrontato nel paragrafo successivo.
Attraverso il seguente esempio verrà illustrato il funzionamento della posta elettronica.
Supponiamo che Alice (alice@topolino.it) voglia inviare un messaggio a Bob (bob@paperino.it).

Alice avrà composto il messaggio utilizzando ad esempio Outlook, e lo invierà al proprio mail server 
sfruttando il protocollo SMTP. Una volta ricevuto il messaggio, il mail server di Alice si incaricherà di 
inoltrare quest'ultimo verso il server di posta di Bob (generalmente non sono utilizzati server di posta intermedi, 
ma le connessioni avvengono direttamente tra i server di posta finali). 
Una volta ricevuta l'email, il server di posta di Bob la immagazzinerà nella sua casella di posta. 
Successivamente quando Bob controllerà la sua mail, il messaggio di posta verrà scaricato dall'user agent 
utilizzato da Bob, si ipotizzi Thunderbird, rendendolo disponibile alla sua vista.
La figura \ref{architetturaPosta} riassume i vari componenti che partecipano al trasferimento del messaggio, dal
mittente al destinatario, e i relativi protocolli utilizzati \cite{kurose2008reti}.

\begin{figure}[htp]
    \centering
    \includegraphics[width=12cm, height=20cm, keepaspectratio]{architettura_posta.jpg}
        \caption{Architettura della posta elettronica.}\label{architetturaPosta}
  \end{figure}


\subsection{Protocollo SMTP}
Come già detto il protocollo SMTP è il principale protocollo utilizzato per la posta elettronica.
SMTP è un protocollo di livello applicativo e utilizza come protocollo di livello di trasporto TCP,
che offre il servizio di consegna dati affidabile. 
L'utilizzo di TCP è principalmente motivato dal fatto che la posta elettronica è un servizio asincrono, 
e non in tempo reale, per questo motivo tollerante a ritardi, mentre non può tollerare errori di trasmissione 
che potrebbero comportare l'alterazione/corruzione del messaggio. 
È utilizzato principalmente per lo scambio tra user agent del mittente e mail server del mittente, e tra mail server 
in generale. È un protocollo di tipo push, e questo significa che la connessione viene aperta da chi vuole 
trasferire i dati, ovvero da chi vuole inviare il messaggio. Dato che SMTP utilizza connessioni persistenti, 
se Alice dovesse inviare a Bob più messaggi di posta, potrebbe inviarli sulla stessa connessione TCP e 
infine chiuderla.
Di seguito un esempio di comunicazione SMTP:
\pagebreak
\begin{verbatim}
    S: 220 smtp.reply.it ESMTP Postfix
    C: HELO smtp.topolino.it
    S: 250 mail.paperino.it
    C: MAIL FROM: <alice@topolino.it>
    S: 250 OK
    C: RCPT TO: <bob@paperino.it>
    S: 250 OK
    C: DATA
    S: 354 End data with <CR><LF>.<CR><LF>
    C: From: "Alice" <alice@topolino.it>
    C: To: "Bob" <bob@paperino.it>
    C: Date: Tue, 15 May 2021 16:02:43 -0500
    C: Subject: Messaggio di prova
    C: 
    C: Ciao, questo è un messaggio di prova
    C: .
    S: 250 Ok: queued as 12345
        C: QUIT
    S: 221 Bye
\end{verbatim}\cite{SMTP}.

\subsection{Protocolli di accesso alla posta: POP3, IMAP, HTTP}
Come detto precedentemente SMTP è un protocollo di tipo push, 
utilizzato per l’invio di posta elettronica. L’user agent di un destinatario non potrà utilizzare questo 
protocollo per scaricare la posta dal suo mail server. 
Per compiere questa operazione è necessario utilizzare dei protocolli di tipo pull. 
Nei protocolli di tipo pull, la connessione TCP viene aperta da chi vuole ricevere i dati. 
L’user agent di un destinatario di posta utilizzerà principalmente IMAP per scaricare la posta in locale 
dal suo mail server (POP3 ha delle limitazioni). 
Molte persone però accedono alla posta elettronica utilizzando il proprio browser (es. Mozilla Firefox), 
in questo caso il protocollo utilizzato per scaricare i messaggi di posta è HTTP.

\subsection{Protocolli crittografici TLS/SSL}
SSL (secure sockets layer) permette di rendere sicuro TCP,  fornendogli servizi di sicurezza, 
comprese la riservatezza, l’integrità dei dati e l’autenticazione del server. 
Una versione particolare di SSL (SSLv3) è chiamata TLS (transport layer security).\cite{tls}

\begin{enumerate}
    \item \textbf{riservatezza}: il traffico viaggia cifrato;
    \item \textbf{integrità dei dati}: se non venisse assicurata l'integrità dei dati, un soggetto terzo
    potrebbe modificare il messaggio;
    \item \textbf{autenticazione da parte del server}: L'autenticazione TLS è unilaterale, solo il server si 
    autentica presso il client. Il mail client valida il certificato del server, controllando che la firma dei 
    certificati del server sia valida e riconosciuta da una certificate authority.
    In questo modo il mail client è sicuro dell'autenticità del mail server. 
    %Per questo motivo Postfix ha bisogno di un certificato da mostrare ai mail client.
\end{enumerate}


